\documentclass{article}

\usepackage[T1]{fontenc}
\usepackage[polish]{babel}
\usepackage[utf8]{inputenc}
\usepackage{titlesec}
\usepackage{titling}
\usepackage[margin=.7in]{geometry}
% \usepackage[margin=1.25in]{geometry}
\usepackage{lastpage}
\usepackage{fancyhdr}
\usepackage{graphicx}
\usepackage{wrapfig}
\usepackage{blindtext}
\usepackage{tcolorbox}
\usepackage{polski}
\usepackage{multicol}
\usepackage{transparent}
\usepackage{ragged2e}
\usepackage{geometry}

\geometry{
  bottom=3cm
%   footskip=2cm
}



\pagestyle{fancy}
\fancyhf{}
\renewcommand{\headrulewidth}{0pt}
\cfoot{\transparent{0.4} \justify{Wyrażam zgodę na przetwarzanie moich danych osobowych dla potrzeb niezbędnych do realizacji procesu rekrutacji (zgodnie z Ustawą z dnia 29.08.1997 roku o Ochronie Danych Osobowych; tekst jednolity: Dz. U. 2016 r. poz. 922).}}
% \chead{}


\titleformat{\section}
{\huge}
{}
{0em}
{\bfseries}[\titlerule\vspace{-.2cm}]

\titleformat{\subsection}
{\vspace{1em}\bfseries\large}
% {\hspace{-.25in}$\bullet$}
{}
{0em}
{}

\titleformat{\subsubsection}[runin]
{\bfseries}
{}
{0em}
{}[\hspace{1em}---]

\titlespacing{\subsubsection}
{0em}{.25em}{1em}

\renewcommand{\maketitle}{
    \begin{center}
        % {\Huge\bfseries\thetitle}

        % \vspace{1.5em}


        {\huge\bfseries\theauthor}

        \vspace{.25em}

        +48 888 829 696\\czajka.adam147@gmail.com\\github.com/czajrow
	\end{center}
	\vspace{-.5cm}
}


\begin{document}


\title{CV}
\author{Adam Czajka}

\maketitle

\vspace{2em}

\begin{minipage}[t]{.5\textwidth}

	\section{edukacja}
	\subsection{Politechnika Warszawska}
	\texttt{10/2017 - obecnie}
	\par Informatyka, Wydział Elektryczny, studia inżynierskie.

	\subsection{John von Neumann University, Węgry}
	\texttt{9/2019 - 5/2020}
	\par Informatyka, wymiana \texttt{Erasmus+}, studia międzynarodowe.

	\vspace{2em}

	\section{doświadczenie zawodowe}

	\subsection{Biuro Informatyki w Urzędzie Mazowieckim}
	\texttt{7/2019 - 8/2019}
	\par Staż w Biurze Informatyki - praca przy odbiorze modułu danych otwartych na stronę \texttt{www.gov.pl}.

	\subsection{Erasmus Student Network}
	\texttt{11/2017 - obecnie}
	\par W ramach mojej działalności w \texttt{ESN} zajmowałem się organizacją wydarzeń dla studentów międzynarodowych w Warszawie. W szczególności przez dwa lata pełniłem funkcję koordynatora sportu \texttt{ESN}.

	\subsection{Mały Inżynier}
	\texttt{6/2018 - 2/2019}
	\par\texttt{Mały Inżynier} to firma w której prowadziłem zajęcia edukacyjne z programowania, eksperymentów, robotyki i majsterkowania dla dzieci.

	\subsection{Europejska Szkoła Korepetycji}
	\texttt{10/2017 - 6/2019}
	\par W \texttt{Europejskiej Szkole Korepetycji} prowadziłem indywidualne lekcje z matematyki i fizyki dla uczniów Liceum przygotowujących się do egzaminu maturalnego z tych przedmiotów.

\end{minipage}\hfill
\begin{minipage}[t]{.4\textwidth}

	\section{realizowane projekty}
	\subsection{Android Studio}
	Aplikacja w technologii Android (Java), w której sieć neuronowa typu \verb+MLP+ (również zaimplementowana własnoręcznie w języku \verb+Java+) prowadzi rozgrywkę z użytkownikiem w grę \verb+Kółko i Krzyżyk+.

	\subsection{JavaFX}
	Aplikacja \verb+JavaFX+ wykorzystująca bibliotekę \verb+OpenCV+ do rozpoznawania pojazdów na filmie.

	\subsection{JavaScript}
	Wizualizacja metody \verb+Monte Carlo+ do obliczania całki oznaczonej dowolnej funkcji matematycznej o rzeczywistej dziedzinie. Wykorzystana została biblioteka \verb+p5.js+.

	\vspace{1em}

	\section{umiejętności}
	\subsection{Technologia}
	Java 8, C, MatLab, Angular 2+, TypeScript, JavaScript, HTML, CSS, Python, bash/zsh, git~(także w wierszu poleceń), vim, \LaTeX, SQL.

	\subsection{Inne}
	podstawowa znajomość \verb+UML+, znajomość podstawowych wzorców projektowych, znajomość zasad \verb+OOP+, znajomość założeń metodyki \verb+SCRUM+.

	\subsection{Języki obce}
	\subsubsection{Angielski} C1 (słownictwo informatyczne)

	\vspace{1em}

	\section{zainteresowania}
	sporty wodne - multimedalista mistrzostw polski w wioślarstwie, były członek kadry narodowej w wioślarstwie, żeglarz morski.

\end{minipage}

\end{document}
